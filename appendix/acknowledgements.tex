% !Mode:: "TeX:UTF-8"

\titlecontents{chapter}[2em]{\vspace{.5\baselineskip}\xiaosan\song}
             {\prechaptername\CJKnumber{\thecontentslabel}\postchaptername\qquad}{} 
             {}                            % 设置该选项为空是为了不让目录中显示页码
\fancypagestyle{plain}   % 设置页眉页脚风格,按照教务处规定,此处出现页眉,但是没有页脚(页码)。
\lhead{}
\rhead{}
\chead{\song\wuhao 最优化方法应用报告} % 设置页眉内容
\lfoot{}
\cfoot{}
\rfoot{}
\markboth{致\quad 谢}{致\quad 谢}
\addcontentsline{toc}{chapter}{致\quad 谢} % 添加到目录中
\chapter*{致\quad 谢}
\setcounter{page}{1}

在这篇报告完成的过程中,首先,我要深深感谢我的导师,[导师的姓名],他们在整个研究过程中提供了坚定的支持、指导和宝贵的见解。他们的专业知识和无私奉献对于塑造这项研究的方向起到了关键作用。

我也要衷心感谢研究团队的成员以及[机构/部门名称]的同事们,他们的协作精神和知识交流丰富了这个项目。他们的贡献对于我们研究目标的达成至关重要。

我要感谢[资助机构/基金会名称]提供的资金支持。这种支持在使我能够有效开展这项研究方面发挥了至关重要的作用。

我要感谢[机构或实验室名称]提供的研究设施和资源,使我能够顺利进行实验和数据收集工作。

对于我的亲朋好友,我要感激不尽,他们在整个研究过程中的鼓励和理解是无法估量的。他们在这段旅程中的支持对我来说是无价的。

我还要感谢匿名审稿人和同行们,他们提供了宝贵的反馈和建设性的批评,极大地提高了这项工作的质量。

最后,我要由衷感谢那些名字可能没有出现在这里的人,但以各种方式为这项研究工作的实现做出了贡献。

感谢大家的支持和鼓励。

[您的姓名]
[您的学术职称(如果适用)]
[您的机构]
[日期]





